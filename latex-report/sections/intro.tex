\section{Introduction}
\subsection{Brief Introduction}
For this project we applied basic statistical methods to the 2019-2020 Democratic Primary election cycle. An election for the President of the United States happens every four years and is the largest political event in the USA. Leading up to the general election, most of the candidates running for president go through a series of state primaries and caucuses. Although these primaries and caucuses are ran differently, they both serve the same purpose. They allow every state to choose their major political parties’ nominees for the general election\cite{usagov}. The campaigning process lasts at least a year for most candidates and costs many millions of dollars in advertising outreach, and travelling. Since the Republican Presidential Nominee is almost guaranteed to be President Donald Trump, we focused on the Democratic Nominee Primaries. Below you will find the list of objectives used in this project:

\subsection{Objectives}
\begin{itemize}
    \item To monitor polling data for the Democratic Party primary candidates
    \item To read the polling data from source into a R dataframe, filter and plot polling data over time and to run analysis and prediction tools using R script 
    \item To plot the fundraising for Democratic candidates
    \item To plot the number of delegates awarded for candidates (state wise)
\end{itemize}

% \subsection{Dataset}
% \begin{itemize}
%     \item Dataset from fivethirtyeight.com\cite{jr._malone_skelley_koerth_paine_dubin_sawchik_rakich} is used in this project and website tracks a large number of different polls
%     \item  The data is available in a csv format and updated on a regular basis. 
%     \item The csv file includes columns like pollster name, state, sponsor, sample size, source, poll start and end dates and poll results as a percentage of votes for each candidate
% \end{itemize}
