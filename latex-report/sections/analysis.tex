\section{Analysis}\label{analysis}

\subsection{Delegates}

Figure \ref{Delegate} shows the total number of delegates gathered by each of the major candidates, grouped by state. It is clear based on this graph alone that Biden and Sanders seem to be the two main candidates that have a chance at winning the Democratic Presidential Nomination. Biden appears to be more successful in south eastern states such as North Carolina, South Carolina, and Alabama, while Sanders tends to be more successful in west and mid-western states such as California, Nevada, and Colorado.

\subsection{Polling}

Looking at figure \ref{Updated-Polling-Data-1}, you will see the average candidate polling support over time. The data is from all A-rated polls collected by FiveThirtyEight. According to this graph, Senator Sanders leads the way. His average support is the highest. Joe Biden's support varies more and is ranked second most of time. Bloomberg, Buttigieg, and Warren are all are fighting for third place, but none of the three clearly come out on top. Pete Buttigieg is one of the youngest candidates, and his achievements in his hometown really changed that city. Looking at poll results however, we can see that his influence and movement is not greatly affecting people outside his city. Starting from middle of February, his support steadily decreased. Comparing with recent news, the graph is correct. Buttigieg, Warren, and Bloomberg have all dropped out. Biden and Sanders will compete for the last position.

Figure \ref{A-rated-polls} contains the same A-rated polling data as figure \ref{Updated-Polling-Data-1}, but is focused around the various primary and caucus events leading up to super tuesday. For most of the time series displayed, Senator Sanders is ranked first in support. Although he lost some support following the 9th Presidential Debate, the polls bounced back leading up to the South Carolina Primary. Joe Biden's support rate dropped starting from Feb. 25 following the widely televised "dog-faced pony tail soldiers" insult that he spoke to a college girl. However support his rate rebounded at Feb. 29 following the South Carolina Primary. During this time frame Bloomberg trades blows with Biden before settling in around third place with Warren. Finally, Buttigieg's support appears to steadily decrease before sharply declining, leading up to him dropping out of the race on Feb. 29. Note that at the time of writing for this report, there was no data available for A-rated polls on super tuesday.

Due to the lack of polling data from super tuesday, we also plotted the averages of all polls, regardless of their rating, which you can see in figure \ref{All-rated-polls}. This gives us a little more data, showing poll results up to super tuesday. In this graph we can see that Biden overtakes Sanders in average support right before super tuesday. Again Bloomberg and Warren are tied in third place, with Buttigieg coming last.

Figures \ref{scatter-A-rated}, \ref{scatter-all}, and \ref{scatter-all-linear} display the polling data as a scatter plot with regression lines. The shaded grey areas represent the $95\%$ confidence interval's for each candidate. These plots only contain the the primary candidates left in the race at the time of analysis. As you can see from the scatter plots, there is a large variation in polling results, even among the A-rated polls. This large variation makes accurate predictions very difficult. In addition, there is a difference in trends between figures \ref{scatter-A-rated} and \ref{scatter-all}. In figure \ref{scatter-A-rated}, only the A-rated polls are plotted. Here it appears that both Sanders and Biden's level of support is declining after the South Carolina Primary. In contrast, figure \ref{scatter-all} contains polling data from all polls and shows the opposite trend. Now both Sanders and Biden show an increasing level of support leading up to and after the South Carolina Primary, with Biden's support increasing faster than Sanders. Figure \ref{scatter-all-linear} shows the same set of polls but displays a linear regression. This graph shows the same general trend as figure \ref{scatter-all}, where both Sanders and Biden have an ever increasing level of support. The discrepancies between polls make accurate predictions based on simple regression lines near impossible.

\subsection{Spending}

Next we will look at the ad spending from each candidate and see if it impacted their level of support. Figure \ref{Updated-Polling-Data-1} shows that prior to Iowa caucus, Sanders has about $25\%$ percent support and he spent around 6.3 million on ads there (figure \ref{IOWA}). His support dropped around $5\%$ before the caucus. After the caucus, his support steadily increased, showing that this event did help him regain traction. Pete also spent about 6.8 million, and his support did raise by about $10\%$ after Iowa caucus. Biden spent around 3.2 million but his support drops around $7\%$ after the Iowa caucus. Warren spent about 4 million and her supports dropped around $4\%$. Based on this, we would conclude that the candidates who spent more money did get a little more gain from the ads than those who spent less. Bloomberg is an outlier because he spent just under 0.7 million but his support after the Iowa caucus only dropped by $1\%$.

In New Hampshire (figure \ref{Newhampshire}), Joe Biden spent just 0.1 million in ads, however his support was increased close to Bernie Sander's who spend around 1 million in the same state to stay at $30\%$ support. Bloomberg spent about 0.15 million in ads, and his support raised up to $25\%$. Warrens support however drops about $8\%$ even though she spent about 0.73 million in ads, which is significantly more than Bloomberg and Biden. 

Finally, in Nevada (figure \ref{Nevada}), Biden spent 0.7 million in ads but his support drops by $5\%$. Bloomberg only spent 0.027 million dollars here, and his support also drops. Sanders spent almost 1 million dollars in Nevada only for his support to drop by $5\%$ before rebounding. Warren spent about 0.84 million dollars on ads in Nevada and her support increases slightly afterwards. From this data we can conclude that ad spending tends to increase support for candidates who spend lots of money on ads, but it is not guaranteed.

\subsection{Funding}

In this last section we will look at funding for each candidate and see if it impacted their level of support. Figure \ref{Total} shows the total funding gathered by each candidate in millions of dollars. Both Steyer and Bloomberg have gathered more than 200 million dollars of funding for their campaigns, greatly overshadowing all other candidates. Since Steyer and Bloomberg have both since dropped out of the race due to very poor performance in the primaries, we can infer that that the large source of funding did not greatly improve their general level of support. Figures \ref{Small Donors} through \ref{Others} break down this total funding into different sources. Steyer and Bloomberg are both multi-billionaires and thus the majority of their funding comes from self funding. Sanders comes in third place at 108 million dollars of total funding, the majority of which came from small donors. Both Warren and Buttigieg gathered more funding for their campaigns than Biden, however they have both since dropped out of the race. Biden and Sanders are the two remaining candidates in the race. Out of the two, Biden has collected significantly less funds than Sanders and yet he is performing very competitively. There does not seem to be a direct connection between the level of funding and the level of support for a candidate.
