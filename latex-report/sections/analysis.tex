\section{Analysis}\label{analysis}

\subsection{Analysis of Delegates}

Figure \ref{Delegate} shows the total number of delegates gathered by each of the major candidates, grouped by state.

\subsection{Analysis of Polling Data}

Looking at figure \ref{Updated-Polling-Data-1}, you will see the average candidate polling support over time. The data is from all A-rated polls collected by FiveThirtyEight. According to this graph, Senator Sanders leads the way. His average support is the highest. Joe Biden's support varies more and is ranked second most of time. Bloomberg, Buttigieg, and Warren are all are fighting for third place, but none of the three clearly come out on top. Pete Buttigieg is one of the youngest candidates, and his achievements in his hometown really changed that city. Looking at poll results however, we can see that his influence and movement is not greatly affecting people outside his city. Starting from middle of February, his was support steadily decreased. Comparing with recent news, the graph is correct. Buttigieg, Warren, and Bloomberg have all dropped out. Biden and Sanders will compete for the last position.

Figure \ref{A-rated-polls} contains the same A-rated polling data as figure \ref{Updated-Polling-Data-1}, but is focused around the various primary and caucus events leading up to super tuesday. For most of the time series displayed, Senator Sanders ranked first in support. Although he lost some support following the 9th Presidential Debate, the polls bounced back leading up to the South Carolina Primary. Joe Biden's support rate dropped starting from Feb. 25 following the widely televised "dog-faced pony tail soldiers" insult that he spoke to a college girl. However his rate rebounded at Feb. 29 following the South Carolina Primary. During this time frame Bloomberg trades blows with Biden before settling in around third place with Warren. Finally, Buttigieg's support appears to steadily decrease before sharply declining, leading up to him dropping out of the race on Feb. 29. Note that at the time of writing for this report, there was no data available for A-rated polls on super tuesday.

Due to the lack of polling data from super tuesday, we also plotted the averages of all polls, regardless of their rating, which you can see in figure \ref{All-rated-polls}. This gives us a little more data, showing poll results up to super tuesday. In this graph we can see that Biden overtakes Sanders in average support right before super tuesday. Again Bloomberg and Warren are tied in third place, with Buttigieg coming last.