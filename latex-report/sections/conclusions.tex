\section{Conclusions}\label{conclusions}

In this project, we focused on applying statistical methods to try and foresee the results of 2019-2020 Democratic Primary election cycle. First, we collected data from FiveThirtyEight and then used the R language to process that data. We generated the plots and established models from different aspects to try to reveal the real situation in the Democratic Primary's. We not only concentrated on the number of delegates and supporting rate of candidates, but also investigated the ad spending and funding of candidates.

From the models we built, it’s obvious that Bernie Sanders and Joe Biden are ranked first and second most often. Bloomberg, Warren and Buttigieg somehow have advantages at one specific point, but these advantages did not last long. Combining all the graphs and plots we have from our model, it is clear that Bernie Sanders and Joe Biden will stand out.

According to recent news, Bernie Sanders and Joe Biden have huge advantages over other candidates. Warren, Bloomberg and Buttigieg have since dropped out. They quit for various reasons, but the main reason is that they had virtually no chance in beating Biden or Sanders to nomination. 

Our model represents today’s situation, however in the political landscape things can change very quickly. From our model, it looks like the future election competition between Biden and Sanders will be very fierce. From the polls that we analysed, both Sanders and Biden have the same tendency for increasing support over the last couple months. Although Sanders has been polling higher than Biden, this doesn’t mean he will be the one who can compete with President Trump. We have yet to get solid results back from super tuesday, which may have a large impact on the polls.

We found that ad spending may have an impact on how a candidate polls, but differences of funding between the top eight candidates did not seem to correlate with their performance. We attempted to do some basic predictions using regression lines, but the large variance in polls coupled with  unpredictable events that can create large swings in support make predictions unreliable.
