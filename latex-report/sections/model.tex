
\section{Model}\label{model}
\subsection{Delegates}
Figure \ref{Delegate} shows the number of delegates awarded after each primary/caucus per candidate. The results are displayed for each state until Super Tuesday. Biden is awarded with the most number number of delegates compared to other candidates. Buttigieg acquired the least number of delegates. 
\begin{figure}[H]
    \centering
    \includegraphics[width=0.9\textwidth]{figures/Delegate.png}
    \caption{Delegate}
    \label{Delegate}
\end{figure}

\subsection{Polling}
The poll data is collected and plotted for top five democratic candidates(Biden, Sanders, Warren, Bloomberg and Buttigieg). Data is plotted for both A rated and for all the polls as displayed in figures \ref{Updated-Polling-Data-1}, \ref{A-rated-polls} and \ref{All-rated-polls}. 
\begin{figure}[H]
    \centering
    \includegraphics[width=0.9\textwidth]{figures/long-A-rated-polls.png}
    \caption{Updated Polling Data}
    \label{Updated-Polling-Data-1}
\end{figure}
Figure \ref{A-rated-polls} displays the support for candidates collected from A rated polls. At the end of  South Carolina Primary , it is observed that Sanders has the highest percentage of support is the highest and Buttigieg has the lowest percentage of support.  
\begin{figure}[H]
    \centering
    \includegraphics[width=0.9\textwidth]{figures/A-rated-polls.png}
    \caption{A Rated Polls}
    \label{A-rated-polls}
\end{figure}

Figure \ref{All-rated-polls} shows support for candidates until Super Tuesday from all rated polls. Biden's support shows a drop, making Biden the top candidate with highest percentage of support and Buttigieg with the lowest support percentage. 
\begin{figure}[H]
    \centering
    \includegraphics[width=0.9\textwidth]{figures/All-rated-polls.png}
    \caption{All Rated Polls}
    \label{All-rated-polls}
\end{figure}

Figure \ref{scatter-A-rated} shows support for candidates until South Carolina Primary. A slight drop is observed in Sanders' percentage of support after Nevada Caucus. Though there's a steady increase in Warren's support after Nevada caucus and South Carolina Primary, she still has the lowest percentage of support. 
\begin{figure}[H]
    \centering
    \includegraphics[width=0.9\textwidth]{figures/scatter-A-rated.png}
    \caption{Scatter Plot for A Rated Polls}
    \label{scatter-A-rated}
\end{figure}

Figure \ref{scatter-all} shows support for candidates until Super Tuesday. There is a steady increase in Sanders' percentage of support after South Carolina Primary. There is a sharp increase in Biden's percentage of support after South Carolina Primary,almost on par with Sanders. Warren's support has been consistently the lowest percentage of support compared to other candidates. 
\begin{figure}[H]
    \centering
    \includegraphics[width=0.9\textwidth]{figures/scatter-all.png}
    \caption{Scatter Plot for All Rated Polls}
    \label{scatter-all}
\end{figure}
Figure \ref{scatter-all-linear} shows the linear trend for candidate support until Super Tuesday. Overall, starting from Iowa Caucus to Super Tuesday, Sanders' support increases steadily whereas a progressive growth for support is observed in Biden's support percentage. Warren's support percentage shows a slight dip overtime.
\begin{figure}[H]
    \centering
    \includegraphics[width=0.9\textwidth]{figures/scatter-all-linear.png}
    \caption{Scatter Plot for All Rated Polls - Linear}
    \label{scatter-all-linear}
\end{figure}

\subsection{Spending}
Figures \ref{MoneyspendinAds} shows the total amount of money spent on advertisements. The highest amount is spent by Bloomberg on ads. However, after the Iowa caucus, there is a dip in Bloomberg's percentage of support, eventually dropping out of the Presidential Election 2020. 

\begin{figure}[H]
    \centering
    \includegraphics[width=0.9\textwidth]{figures/MoneyspendinAds.png}
    \caption{Money spending on Ads}
    \label{MoneyspendinAds}
\end{figure}

Figures \ref{IOWA} shows the amount of money spent on advertisements in Iowa. The highest amount is spent by Stayer in Iowa. However, Steyer ended up with 0 delegates after the Iowa caucus. 
\begin{figure}[H]
    \centering
    \includegraphics[width=0.9\textwidth]{figures/IOWA.png}
    \caption{Money spending on Ads in Iowa}
    \label{IOWA}
\end{figure}
Figures \ref{Newhampshire} shows the amount of money spent on advertisements in New Hampshire. The highest amount is again spent by Stayer and he  ended up with 0 delegates after the New Hampshire primary. 
\begin{figure}[H]
    \centering
    \includegraphics[width=0.9\textwidth]{figures/Newhampshire.png}
    \caption{Money spending on Ads in New Hampshire}
    \label{Newhampshire}
\end{figure}
Figures \ref{Nevada} shows the amount of money spent on advertisements in Nevada. The highest amount is spent by Stayer, again ending up with 0 delegates after the Nevada caucus. 
\begin{figure}[H]
    \centering
    \includegraphics[width=0.9\textwidth]{figures/Nevada.png}
    \caption{Money spending on Ads in Nevada}
    \label{Nevada}
\end{figure}

\subsection{Funding}
Figure \ref{Total} shows the total funding for presidential elections. Steyer is funded with maximum amount and Gabbard with the least amount for the elections. 
\begin{figure}[H]
    \centering
    \includegraphics[width=0.9\textwidth]{figures/Total.png}
    \caption{Total Funding}
    \label{Total}
\end{figure}

Figure \ref{Small Donors} shows the funding from small donors for presidential elections. Sanders is funded with maximum amount and Bloomberg with the least amount from small donors for the elections. 
\begin{figure}[H]
    \centering
    \includegraphics[width=0.9\textwidth]{figures/Small Donors.png}
    \caption{Small Donors Funding}
    \label{Small Donors}
\end{figure}

Figure \ref{Bigdonor} shows the funding from big donors for presidential elections. Buttigieg has maximum funds from big donors and Bloomberg again with the least funding for the elections. 
\begin{figure}[H]
    \centering
    \includegraphics[width=0.9\textwidth]{figures/Bigdonor.png}
    \caption{Big donors Funding}
    \label{Bigdonor}
\end{figure}

Figure \ref{Selffunnded} shows self funding for presidential elections. Bloomberg and Steyers has maximum self funding. Rest of the candidates have the least or zero self funding. 
\begin{figure}[H]
    \centering
    \includegraphics[width=0.9\textwidth]{figures/Selffunnded.png}
    \caption{Self funded Funding}
    \label{Selffunnded}
\end{figure}

Figure \ref{Transfer} shows transfer funding for presidential elections. Sanders has maximum transfer funding followed by Warren. Steyers, Bloomberg and Buttigieg  have the least transfer funding. 
\begin{figure}[H]
    \centering
    \includegraphics[width=0.9\textwidth]{figures/Transfer.png}
    \caption{Transfer Funding}
    \label{Transfer}
\end{figure}
Figure \ref{Others} shows transfer funding for presidential elections. Steyers has maximum funding from other sources followed by Buttigieg. Klobuchar has the least funding from other sources. 
\begin{figure}[H]
    \centering
    \includegraphics[width=0.9\textwidth]{figures/Others.png}
    \caption{Others Sources Funding}
    \label{Others}
\end{figure}
