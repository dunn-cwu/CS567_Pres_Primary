\section{Design}\label{design}

\subsection{Data Set}

All of the polling data used in our analysis was downloaded from FiveThirtyEight, a well known political blog that does predictions for various elections in the USA\cite{jr._malone_skelley_koerth_paine_dubin_sawchik_rakich}. They collect polling data from a large number of different pollsters and then combine all into a single $.csv$ file that is available for download. This data set is updated on a regular basis with new polling results, and contains many different columns such as pollster name, state, sponsor, sample size, source, poll start and end dates, and poll results as a percentage of votes for each candidate. In addition, FiveThirtyEight rates all pollsters based on their quality and level of bias. Ratings range from A+ down to F. Pollsters with the least bias have the best ratings. Using their ratings, we are able to filter the data set such that only highly rated polls are considered. Other than polling data, FiveThirtyEight presents many other aspects of the election we can analyze such as candidate funding and number of delegates.

\subsection{Implementation}

For our experiments we implemented a program in R script that can import the data set from FiveThirtyEight and generate various plots of the polling data. Using these plots we can visualize the data and conduct some analysis, looking for trends. Figure \ref{code:filter} below contains an example code snippet that shows how we filter out the important features from the data set.

\begin{figure}[H]
\centering
\begin{lstlisting}[language=R]
# Load data set from csv file
pdata<-read.csv("data/president_primary_polls.csv", header = TRUE)

# Filter out democrats and needed columns
pdata_dem<-pdata[pdata$party=="DEM", 
                 c("question_id", "poll_id", "pollster_id", 
                   "fte_grade", "end_date", "candidate_id", 
                   "candidate_name", "pct")]

# Replace blank cells with NA
pdata_dem[pdata_dem == ""] = NA

# Replace data strings with date data type
pdata_dem$end_date <- as.Date(pdata_dem$end_date, format = "%m/%d/%y")

# Filter out date range
pdata_dem<-pdata_dem[pdata_dem$end_date >= "2020-01-01",]

# Sort dataframe by poll end date, question id, and candidate id
pdata_dem<-pdata_dem[order(pdata_dem$end_date, 
                     pdata_dem$question_id, pdata_dem$candidate_id),]

# Filter out all polls that are rated worse than A-
pdata_dem<-subset(pdata_dem, fte_grade == 'A+' | 
                    fte_grade == 'A' | fte_grade == 'A-')
\end{lstlisting}
\caption{Example Code Snippet - Data Filtering}
\label{code:filter}
\end{figure}
